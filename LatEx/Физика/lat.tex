\documentclass[8pt]{book}
\usepackage{graphicx}
\usepackage{float}
\usepackage[utf8x]{inputenc}
\usepackage[english,russian]{babel}

\begin{document}

$p= \frac{1}{3}m_0nv_{KB}^{2} = \frac{2}{3}nE_k=nkT $ 

$v_{KB}= \sqrt \frac{3RT}{M} $ 

$ pV= \nu RT \vert k=\frac{R}{N_a}$

$ \eta = 1 - \frac{T_2}{T_1}  $

$ C_V = \frac{iR}{2} $

\begin{table}[H] 
\caption{\label{tab:summary}Процессы в цикле Карно(2 изо и 2 адиб)} 
\begin{center} 
\begin{tabular}{|c|c|} 
\hline 
Процес & Рабта \\ 
\hline 
Изотерма & $ A_{12}=\frac{m}{\mu}RT_1\ln{\frac{V_2}{V_1}}=Q_1$ \\ 
$ T=const; V_2>V_1; U=0 $ \\
\hline 
Изобара &  $ Q = \Delta{U} + A= \frac{i}{2}A+p\delta{V} $ \\
$ p = const $ \\
\hline 
Изахора &  $ Q = \Delta{U}= \frac{i}{2}p\delta{V} $ \\
$ V = const; A=0 $ \\
\hline
Адиабата &  $ A_{23} = -\frac{m}{\mu}C_V(T_2-T_1) $ \\
$ \Delta{Q}=0 $ \\
\hline  
\multicolumn{2}{|c|}{Общая работа цикла} \\ 
\hline 
\multicolumn{2}{|c|}{$A=A_{12}+A_{23}-A_{43}-A_{41}= Q_1-Q_2$ }\\ 
\hline 
\end{tabular} 
\end{center} 
\end{table}


\end{document}
